\documentclass{article}
\usepackage[spanish]{babel}
\usepackage{amsthm}
\usepackage{amssymb}
\theoremstyle{plain}
\newtheorem{theorem}{Teorema }[section]
\newtheorem{corollary}{Corolario}[theorem]
\newtheorem{lemma}[theorem]{Lema}

\begin{document}
\begin{theorem}
Supóngase que \( f \) es continua en \( a \), y \( f(a) > 0 \). Entonces existe un número \( \delta > 0 \) tal que \( f(x) > 0 \) para todo \( x \) que satisface \( |x - a| < \delta \). Análogamente, si \( f(a) < 0 \) entonces existe un número \( \delta > 0 \) tal que \( f(x) < 0 \) para todo \( x \) que satisface \( |x - a| < \delta \).
\end{theorem}
\begin{proof}
Considérese el caso \(f(a)>0\) puesto que \(f\) que es continua en \(a\), si \(\xi>0\) existe un 
\(\delta>0\) tal que, para todo \(x\),
\begin{center}
   si \(|x - a| < \delta\) entonces \(|f(x) - f(a)| < \xi\)
   \end{center}
Puesto que \(f(a)>0\) podemos tomar a \(f(a)\) como el \(\xi\). Así pues, existe \(\delta>0\) tal que para todo \(x\),
\begin{center}
    si \(|x - a| < \delta\), entonces \(|f(x) - f(a)| < f(a)\),
\end{center}
y esta última iguldad implica \(f(x)>0\).
\end{proof}
\end{document}